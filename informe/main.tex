\documentclass[12pt,a4paper]{article}

\usepackage[utf8]{inputenc}
\usepackage[T1]{fontenc}
\usepackage[spanish]{babel}
\usepackage{graphicx}
\usepackage{amsmath, amssymb}
\usepackage{geometry}
\usepackage{hyperref}
\usepackage{enumitem}
\usepackage{booktabs}
\usepackage{array}

\geometry{margin=2.5cm}

\title{Anteproyecto - Proyecto Integrador \\
\large INFB6052 - Herramientas para Ciencia de Datos}
\author{Helen Maureira Barrenechea}
\date{Segundo Semestre 2025}

\begin{document}

\maketitle
\thispagestyle{empty}
\newpage

\section*{Resumen}
Escribir aquí un resumen breve del proyecto.

\newpage
\section{Introducción}
Explicar el problema y la relación con ciencia de datos.

\section{Justificación}
- Relevancia del problema.  
- Impacto esperado.  
- Estado actual.  

\section{Objetivos}
\subsection{Objetivo General}
Escribir aquí.

\subsection{Objetivos Específicos}
\begin{itemize}
    \item Objetivo específico 1
    \item Objetivo específico 2
    \item Objetivo específico 3
\end{itemize}

\section{Metodología}
Explicar fuentes de datos, técnicas y herramientas.

\section{Estado del arte}
Revisión de 3 referencias recientes.  

\section{Cronograma Tentativo}
\begin{tabular}{|p{4cm}|p{10cm}|}
\hline
\textbf{Dia} & \textbf{Actividad} \\
\hline
Dia 1 & Definición del problema \\
\hline
Dia 2 & Limpieza de datos \\
\hline
Dia 3 & Modelado inicial \\
\hline
Dia 4 & Evaluación \\
\hline
Dia 5 & Informe y defensa \\
\hline
\end{tabular}

\section{Referencias}
\begin{itemize}
    \item Autor, A. (2023). \textit{Título}. Revista o fuente.
    \item Autor, B. (2024). \textit{Título}. Conferencia o sitio web.
    \item Autor, C. (2022). \textit{Título}. Disponible en: \url{https://...}
\end{itemize}

\end{document}
